\documentclass[]{article}
\usepackage{lmodern}
\usepackage{amssymb,amsmath}
\usepackage{ifxetex,ifluatex}
\usepackage{fixltx2e} % provides \textsubscript
\ifnum 0\ifxetex 1\fi\ifluatex 1\fi=0 % if pdftex
  \usepackage[T1]{fontenc}
  \usepackage[utf8]{inputenc}
\else % if luatex or xelatex
  \ifxetex
    \usepackage{mathspec}
  \else
    \usepackage{fontspec}
  \fi
  \defaultfontfeatures{Ligatures=TeX,Scale=MatchLowercase}
\fi
% use upquote if available, for straight quotes in verbatim environments
\IfFileExists{upquote.sty}{\usepackage{upquote}}{}
% use microtype if available
\IfFileExists{microtype.sty}{%
\usepackage{microtype}
\UseMicrotypeSet[protrusion]{basicmath} % disable protrusion for tt fonts
}{}
\usepackage[margin=1in]{geometry}
\usepackage{hyperref}
\hypersetup{unicode=true,
            pdftitle={ABO-Gene Allele Frequency Estimation Methods using Phenotypic Blood-Type Data.},
            pdfauthor={Faizan Khalid Mohsin},
            pdfborder={0 0 0},
            breaklinks=true}
\urlstyle{same}  % don't use monospace font for urls
\usepackage{color}
\usepackage{fancyvrb}
\newcommand{\VerbBar}{|}
\newcommand{\VERB}{\Verb[commandchars=\\\{\}]}
\DefineVerbatimEnvironment{Highlighting}{Verbatim}{commandchars=\\\{\}}
% Add ',fontsize=\small' for more characters per line
\usepackage{framed}
\definecolor{shadecolor}{RGB}{248,248,248}
\newenvironment{Shaded}{\begin{snugshade}}{\end{snugshade}}
\newcommand{\KeywordTok}[1]{\textcolor[rgb]{0.13,0.29,0.53}{\textbf{#1}}}
\newcommand{\DataTypeTok}[1]{\textcolor[rgb]{0.13,0.29,0.53}{#1}}
\newcommand{\DecValTok}[1]{\textcolor[rgb]{0.00,0.00,0.81}{#1}}
\newcommand{\BaseNTok}[1]{\textcolor[rgb]{0.00,0.00,0.81}{#1}}
\newcommand{\FloatTok}[1]{\textcolor[rgb]{0.00,0.00,0.81}{#1}}
\newcommand{\ConstantTok}[1]{\textcolor[rgb]{0.00,0.00,0.00}{#1}}
\newcommand{\CharTok}[1]{\textcolor[rgb]{0.31,0.60,0.02}{#1}}
\newcommand{\SpecialCharTok}[1]{\textcolor[rgb]{0.00,0.00,0.00}{#1}}
\newcommand{\StringTok}[1]{\textcolor[rgb]{0.31,0.60,0.02}{#1}}
\newcommand{\VerbatimStringTok}[1]{\textcolor[rgb]{0.31,0.60,0.02}{#1}}
\newcommand{\SpecialStringTok}[1]{\textcolor[rgb]{0.31,0.60,0.02}{#1}}
\newcommand{\ImportTok}[1]{#1}
\newcommand{\CommentTok}[1]{\textcolor[rgb]{0.56,0.35,0.01}{\textit{#1}}}
\newcommand{\DocumentationTok}[1]{\textcolor[rgb]{0.56,0.35,0.01}{\textbf{\textit{#1}}}}
\newcommand{\AnnotationTok}[1]{\textcolor[rgb]{0.56,0.35,0.01}{\textbf{\textit{#1}}}}
\newcommand{\CommentVarTok}[1]{\textcolor[rgb]{0.56,0.35,0.01}{\textbf{\textit{#1}}}}
\newcommand{\OtherTok}[1]{\textcolor[rgb]{0.56,0.35,0.01}{#1}}
\newcommand{\FunctionTok}[1]{\textcolor[rgb]{0.00,0.00,0.00}{#1}}
\newcommand{\VariableTok}[1]{\textcolor[rgb]{0.00,0.00,0.00}{#1}}
\newcommand{\ControlFlowTok}[1]{\textcolor[rgb]{0.13,0.29,0.53}{\textbf{#1}}}
\newcommand{\OperatorTok}[1]{\textcolor[rgb]{0.81,0.36,0.00}{\textbf{#1}}}
\newcommand{\BuiltInTok}[1]{#1}
\newcommand{\ExtensionTok}[1]{#1}
\newcommand{\PreprocessorTok}[1]{\textcolor[rgb]{0.56,0.35,0.01}{\textit{#1}}}
\newcommand{\AttributeTok}[1]{\textcolor[rgb]{0.77,0.63,0.00}{#1}}
\newcommand{\RegionMarkerTok}[1]{#1}
\newcommand{\InformationTok}[1]{\textcolor[rgb]{0.56,0.35,0.01}{\textbf{\textit{#1}}}}
\newcommand{\WarningTok}[1]{\textcolor[rgb]{0.56,0.35,0.01}{\textbf{\textit{#1}}}}
\newcommand{\AlertTok}[1]{\textcolor[rgb]{0.94,0.16,0.16}{#1}}
\newcommand{\ErrorTok}[1]{\textcolor[rgb]{0.64,0.00,0.00}{\textbf{#1}}}
\newcommand{\NormalTok}[1]{#1}
\usepackage{graphicx,grffile}
\makeatletter
\def\maxwidth{\ifdim\Gin@nat@width>\linewidth\linewidth\else\Gin@nat@width\fi}
\def\maxheight{\ifdim\Gin@nat@height>\textheight\textheight\else\Gin@nat@height\fi}
\makeatother
% Scale images if necessary, so that they will not overflow the page
% margins by default, and it is still possible to overwrite the defaults
% using explicit options in \includegraphics[width, height, ...]{}
\setkeys{Gin}{width=\maxwidth,height=\maxheight,keepaspectratio}
\IfFileExists{parskip.sty}{%
\usepackage{parskip}
}{% else
\setlength{\parindent}{0pt}
\setlength{\parskip}{6pt plus 2pt minus 1pt}
}
\setlength{\emergencystretch}{3em}  % prevent overfull lines
\providecommand{\tightlist}{%
  \setlength{\itemsep}{0pt}\setlength{\parskip}{0pt}}
\setcounter{secnumdepth}{0}
% Redefines (sub)paragraphs to behave more like sections
\ifx\paragraph\undefined\else
\let\oldparagraph\paragraph
\renewcommand{\paragraph}[1]{\oldparagraph{#1}\mbox{}}
\fi
\ifx\subparagraph\undefined\else
\let\oldsubparagraph\subparagraph
\renewcommand{\subparagraph}[1]{\oldsubparagraph{#1}\mbox{}}
\fi

%%% Use protect on footnotes to avoid problems with footnotes in titles
\let\rmarkdownfootnote\footnote%
\def\footnote{\protect\rmarkdownfootnote}

%%% Change title format to be more compact
\usepackage{titling}

% Create subtitle command for use in maketitle
\providecommand{\subtitle}[1]{
  \posttitle{
    \begin{center}\large#1\end{center}
    }
}

\setlength{\droptitle}{-2em}

  \title{ABO-Gene Allele Frequency Estimation Methods using Phenotypic Blood-Type
Data.}
    \pretitle{\vspace{\droptitle}\centering\huge}
  \posttitle{\par}
    \author{Faizan Khalid Mohsin}
    \preauthor{\centering\large\emph}
  \postauthor{\par}
      \predate{\centering\large\emph}
  \postdate{\par}
    \date{September 22, 2020}


\begin{document}
\maketitle

\section{Introduction}\label{introduction}

We have a theoretical frame work where we know that pairing combinations
of the 3 alleles (antigens: A, B, O) on the ABO-gene (ABO locus) found
on chromosome 9, lead to four phenotypic manifestations in the form of 4
blood types: A, B, AB, O. Now, there is an interest in estimating the
frequency of the 3 alleles A, B, O in the population. One way to do this
would be to take blood samples of a big sample of people from the
population of interest and using DNA sequencing to determine the allele
type (A or B or O) on the ABO-gene on chromosome 9. However, to get
reliable frequency estimates one needs to do to DNA sequency for a large
number of people, which will be extremely expensive and time consuming.
A more practical and more efficient method for estimating the 3 allele
frequencies of a population would be to collect the phenotypic data, the
blood type A, B, AB or O of people, which would be much more cheaper and
easier to do at a large scale and using the theoretical framework to
estimate the 3 allele frequencies in the population using the phenotypic
blood type data.\\
In this paper we will layout the theoretical framework for estimating
the allele frequency using phenotypic blood type data and use two
separate algorithms and approaches for the estimation. Finally, we will
use the actual blood type data gathered from a sample of 2114 people to
demonstrate and estimate the allele frequency using the two algorithms.

\section{Method}\label{method}

We use two methods for approximating the allele frequencies:
Expectation-Maximization method and Newton-Raphson method.

In the ideal case we could sample n people and find out how many people
possess allele A (nA), how many possess allele B (nB) and how many
possess allele (nO). In such a case it would be very easy to calculate
the allele frequency in the population: freq(A) = nA/n, freq(B) = nB/n,
and freq(O) = nO/n. And we would be done. However, getting nA, nB, nO
directly from DNA sequencing is very expensive and time consuming. It is
much easier to collect the blood type of each individual from the sample
of n people. This will give us a numeric count of how many people have
each blood type, namely, n\_A: number of people with blood type A, n\_B:
number of people with blood type B, n\_AB: number of people with blood
type AB and n\_O: number of people with blood type O. We use genetics to
link the number of individuals with alleles nA, nB, and nO with number
of people with phenotypes n\_A, n\_B, n\_AB, and n\_O (blood-type).

\subsection{Theoretical Framework.}\label{theoretical-framework.}

\subsubsection{Genetical Theory}\label{genetical-theory}

From genetic theory we know that alleles A, B are dominant to allele O;
alleles A, B are co-dominant; and allele O is recessive to alleles A, B.
Using this below is a mapping from genotype to phenotype. Genotype
Phenotype AA or AO - A BB or BO - B AB - AB OO - O

\subsubsection{Hardy--Weinberg
equilibrium.}\label{hardyweinberg-equilibrium.}

Going from allele frequency to genotypic frequency and going from
genotypic frequency to phenotypic frequency.

he Hardy-Weinberg principle states that a population's allele and
genotype frequencies will remain constant in the absence of evolutionary
mechanisms. Ultimately, the Hardy-Weinberg principle models a population
without evolution under the following conditions:

no mutations no immigration/emigration no natural selection no sexual
selection a large population Although no real-world population can
satisfy all of these conditions, the principle still offers a useful
model for population analysis.

Hardy-Weinberg Equations and Analysis According to the Hardy-Weinberg
principle, the variable p often represents the frequency of a particular
allele, usually a dominant one. If p and q are the only two possible
alleles for this characteristic, then the sum of the frequencies must
add up to 1, or 100 percent. We can also write this as p + q = 1. If the
frequency of the Y allele in the population is 0.6, then we know that
the frequency of the y allele is 0.4.

From the Hardy-Weinberg principle and the known allele frequencies, we
can also infer the frequencies of the genotypes. Since each individual
carries two alleles per gene (Y or y), we can predict the frequencies of
these genotypes with a chi square. If two alleles are drawn at random
from the gene pool, we can determine the probability of each genotype.

The Hardy-Weinberg principle assumes that in a given population, the
population is large and is not experiencing mutation, migration, natural
selection, or sexual selection. The frequency of alleles in a population
can be represented by p + q = 1, with p equal to the frequency of the
dominant allele and q equal to the frequency of the recessive allele.
The frequency of genotypes in a population can be represented by
p2+2pq+q2= 1, with p2 equal to the frequency of the homozygous dominant
genotype, 2pq equal to the frequency of the heterozygous genotype, and
q2 equal to the frequency of the recessive genotype. The frequency of
alleles can be estimated by calculating the frequency of the recessive
genotype, then calculating the square root of that frequency in order to
determine the frequency of the recessive allele.

How reasonable is it to assume HWE.

\subsubsection{Statistical Theory:}\label{statistical-theory}

We have our log likelihood function:

We can find the maxima's of this by taking the first derivative,
equating to zero and then solving for p and q. However, this is very
difficult to do analytically. Hence, we will use two different
algorithms and approaches to estimate p and q. Namely, the
Estimation-Maximization algorithm and the Newton-Raphson algorithm.

\subsection{Estimation-Maximization Algorithm
Method}\label{estimation-maximization-algorithm-method}

One method to solving this is to think of the allele frequencies as
latent variables or missing variables. In this approach we can cast the
problem in the framework of EM algorithm.

To implement the EM algorithm we reframe the question in terms of a
missing data problem.

\subsection{Newton-Raphson Algorithm
Method}\label{newton-raphson-algorithm-method}

The other method is a more direct method where we directly find the
maxima's of the loglikelihood function using numeric computational
method called Newton-Raphson algorithm.

To approximate the values of \(\hat{p}, \hat{q}.\)

We have the log-likelihood function

\[ln(L) \sim n_\text{a}\ln\left(p^2+2\left(-p q+1\right)p\right)+n_\text{ab}\ln\left(2qp\right)+2n_\text{o}\ln\left(-p-q+1\right)+n_\text{b}\ln\left(2q\left(-p-q+1\right)+q^2\right)
\]

Using the properties of log we can be simplified to the following:

\[ln(L) \sim n_\text{b}\ln\left(-q\left(2p+q-2\right)\right)+n_\text{a}\ln\left(-p\left(p+2q-2\right)\right)+n_\text{ab}\ln\left(2qp\right)+2n_\text{o}\ln\left(-p-q+1\right)
\]

We need to get the full first derivative of this with respect to p and
q.

We find that the derivative with respect to p is df/dp:

\[ \frac{\partial f}{\partial p } = \dfrac{2n_\text{b}}{2p+q-2}+n_\text{a}\left(\dfrac{1}{p+2q-2}+\dfrac{1}{p}\right)+\dfrac{n_\text{ab}}{p}-\dfrac{2n_\text{o}}{-p-q+1}
\]

The first derivative with respect to q is df/dq:

\[\frac{\partial f}{\partial q } = \dfrac{2n_\text{a}}{2q+p-2}+n_\text{b}\left(\dfrac{1}{q+2p-2}+\dfrac{1}{q}\right)+\dfrac{n_\text{ab}}{q}-\dfrac{2n_\text{o}}{-q-p+1}
\]

The second derivatives which are the components of the hessian are as
follows:

ddfpp

\[ \frac{\partial^2 f}{\partial p^2} = -\dfrac{4n_\text{b}}{\left(2p+q-2\right)^2}+n_\text{a}\left(-\dfrac{1}{\left(p+2q-2\right)^2}-\dfrac{1}{p^2}\right)-\dfrac{n_\text{ab}}{p^2}-\dfrac{2n_\text{o}}{\left(-p-q+1\right)^2}
\]

ddfqp

\[\frac{\partial^2 f}{\partial q \partial p } = -\dfrac{2n_\text{b}}{\left(2p+q-2\right)^2}-\dfrac{2n_\text{a}}{\left(p+2q-2\right)^2}-\dfrac{2n_\text{o}}{\left(-p-q+1\right)^2}
\]

ddfqq

\[ \frac{\partial^2 f}{\partial q^2} = -\dfrac{4n_\text{a}}{\left(2q+p-2\right)^2}+n_\text{b}\left(-\dfrac{1}{\left(q+2p-2\right)^2}-\dfrac{1}{q^2}\right)-\dfrac{n_\text{ab}}{q^2}-\dfrac{2n_\text{o}}{\left(-q-p+1\right)^2}
\]

ddfpq

\[\frac{\partial^2 f}{\partial p \partial q } =-\dfrac{2n_\text{b}}{\left(2p+q-2\right)^2}-\dfrac{2n_\text{a}}{\left(p+2q-2\right)^2}-\dfrac{2n_\text{o}}{\left(-p-q+1\right)^2}
\]

\subsection{Data}\label{data}

\section{Analysis and Results.}\label{analysis-and-results.}

\subsection{Newton-Raphson Algorithm}\label{newton-raphson-algorithm}

\begin{Shaded}
\begin{Highlighting}[]
\CommentTok{# Data}

\NormalTok{n_A =}\StringTok{ }\DecValTok{9123}
\NormalTok{n_B =}\StringTok{ }\DecValTok{2987}
\NormalTok{n_AB =}\StringTok{ }\DecValTok{1269}
\NormalTok{n_O =}\StringTok{ }\DecValTok{7725}
\NormalTok{n =}\StringTok{ }\NormalTok{n_A }\OperatorTok{+}\StringTok{ }\NormalTok{n_B }\OperatorTok{+}\StringTok{ }\NormalTok{n_AB }\OperatorTok{+}\StringTok{ }\NormalTok{n_O}
\end{Highlighting}
\end{Shaded}

\begin{Shaded}
\begin{Highlighting}[]
\CommentTok{# The log likelihood function,}

\NormalTok{loglikf =}\StringTok{ }\ControlFlowTok{function}\NormalTok{(p, q) \{}

\NormalTok{  f =}\StringTok{ }\NormalTok{n_A }\OperatorTok{*}\StringTok{ }\KeywordTok{log}\NormalTok{(p}\OperatorTok{^}\DecValTok{2}\OperatorTok{+}\DecValTok{2}\OperatorTok{*}\NormalTok{p}\OperatorTok{*}\NormalTok{(}\DecValTok{1}\NormalTok{−p−q)) }\OperatorTok{+}\StringTok{ }\NormalTok{n_B}\OperatorTok{*}\KeywordTok{log}\NormalTok{(q}\OperatorTok{^}\DecValTok{2}\OperatorTok{+}\DecValTok{2}\OperatorTok{*}\NormalTok{q}\OperatorTok{*}\NormalTok{(}\DecValTok{1}\NormalTok{−p−q)) }\OperatorTok{+}\StringTok{ }\NormalTok{n_AB}\OperatorTok{*}\KeywordTok{log}\NormalTok{(}\DecValTok{2}\OperatorTok{*}\NormalTok{p}\OperatorTok{*}\NormalTok{q) }\OperatorTok{+}\StringTok{ }\NormalTok{n_O}\OperatorTok{*}\KeywordTok{log}\NormalTok{((}\DecValTok{1}\NormalTok{−p−q)}\OperatorTok{^}\DecValTok{2}\NormalTok{)}

  \KeywordTok{return}\NormalTok{(f)}
\NormalTok{\}}

\CommentTok{# Function that calculates the full first derivative. }

\NormalTok{Df =}\StringTok{ }\ControlFlowTok{function}\NormalTok{(p, q)\{}
  
\NormalTok{  dfp =}\StringTok{ }\NormalTok{(}\DecValTok{2}\OperatorTok{*}\NormalTok{n_B)}\OperatorTok{/}\NormalTok{(}\DecValTok{2}\OperatorTok{*}\NormalTok{p}\OperatorTok{+}\NormalTok{q}\OperatorTok{-}\DecValTok{2}\NormalTok{)}\OperatorTok{+}\NormalTok{n_A}\OperatorTok{*}\NormalTok{(}\DecValTok{1}\OperatorTok{/}\NormalTok{(p}\OperatorTok{+}\DecValTok{2}\OperatorTok{*}\NormalTok{q}\OperatorTok{-}\DecValTok{2}\NormalTok{)}\OperatorTok{+}\DecValTok{1}\OperatorTok{/}\NormalTok{p)}\OperatorTok{+}\NormalTok{n_AB}\OperatorTok{/}\NormalTok{p}\OperatorTok{-}\NormalTok{(}\DecValTok{2}\OperatorTok{*}\NormalTok{n_O)}\OperatorTok{/}\NormalTok{(}\OperatorTok{-}\NormalTok{p}\OperatorTok{-}\NormalTok{q}\OperatorTok{+}\DecValTok{1}\NormalTok{)}
  
\NormalTok{  dfq =}\StringTok{ }\NormalTok{(}\DecValTok{2}\OperatorTok{*}\NormalTok{n_A)}\OperatorTok{/}\NormalTok{(}\DecValTok{2}\OperatorTok{*}\NormalTok{q}\OperatorTok{+}\NormalTok{p}\OperatorTok{-}\DecValTok{2}\NormalTok{)}\OperatorTok{+}\NormalTok{n_B}\OperatorTok{*}\NormalTok{(}\DecValTok{1}\OperatorTok{/}\NormalTok{(q}\OperatorTok{+}\DecValTok{2}\OperatorTok{*}\NormalTok{p}\OperatorTok{-}\DecValTok{2}\NormalTok{)}\OperatorTok{+}\DecValTok{1}\OperatorTok{/}\NormalTok{q)}\OperatorTok{+}\NormalTok{n_AB}\OperatorTok{/}\NormalTok{q}\OperatorTok{-}\NormalTok{(}\DecValTok{2}\OperatorTok{*}\NormalTok{n_O)}\OperatorTok{/}\NormalTok{(}\OperatorTok{-}\NormalTok{q}\OperatorTok{-}\NormalTok{p}\OperatorTok{+}\DecValTok{1}\NormalTok{)}
  
  \KeywordTok{return}\NormalTok{( }\KeywordTok{c}\NormalTok{(dfp, dfq) )}
\NormalTok{\}}

\CommentTok{# Function that calculates the Hessian.}

\NormalTok{DDf =}\StringTok{ }\ControlFlowTok{function}\NormalTok{(p, q) \{}
  
\NormalTok{    d2fpp =}\StringTok{ }\OperatorTok{-}\NormalTok{(}\DecValTok{4}\OperatorTok{*}\NormalTok{n_B)}\OperatorTok{/}\NormalTok{(}\DecValTok{2}\OperatorTok{*}\NormalTok{p}\OperatorTok{+}\NormalTok{q}\OperatorTok{-}\DecValTok{2}\NormalTok{)}\OperatorTok{^}\DecValTok{2}\OperatorTok{+}\NormalTok{n_A}\OperatorTok{*}\NormalTok{(}\OperatorTok{-}\DecValTok{1}\OperatorTok{/}\NormalTok{(p}\OperatorTok{+}\DecValTok{2}\OperatorTok{*}\NormalTok{q}\OperatorTok{-}\DecValTok{2}\NormalTok{)}\OperatorTok{^}\DecValTok{2}\OperatorTok{-}\DecValTok{1}\OperatorTok{/}\NormalTok{p}\OperatorTok{^}\DecValTok{2}\NormalTok{)}\OperatorTok{-}\NormalTok{n_AB}\OperatorTok{/}\NormalTok{p}\OperatorTok{^}\DecValTok{2}\OperatorTok{-}\NormalTok{(}\DecValTok{2}\OperatorTok{*}\NormalTok{n_O)}\OperatorTok{/}\NormalTok{(}\OperatorTok{-}\NormalTok{p}\OperatorTok{-}\NormalTok{q}\OperatorTok{+}\DecValTok{1}\NormalTok{)}\OperatorTok{^}\DecValTok{2}
\NormalTok{    a =}\StringTok{ }\NormalTok{d2fpp}
    
\NormalTok{    d2fqp =}\StringTok{ }\OperatorTok{-}\NormalTok{(}\DecValTok{2}\OperatorTok{*}\NormalTok{n_A)}\OperatorTok{/}\NormalTok{(}\DecValTok{2}\OperatorTok{*}\NormalTok{q}\OperatorTok{+}\NormalTok{p}\OperatorTok{-}\DecValTok{2}\NormalTok{)}\OperatorTok{^}\DecValTok{2}\OperatorTok{-}\NormalTok{(}\DecValTok{2}\OperatorTok{*}\NormalTok{n_B)}\OperatorTok{/}\NormalTok{(q}\OperatorTok{+}\DecValTok{2}\OperatorTok{*}\NormalTok{p}\OperatorTok{-}\DecValTok{2}\NormalTok{)}\OperatorTok{^}\DecValTok{2}\OperatorTok{-}\NormalTok{(}\DecValTok{2}\OperatorTok{*}\NormalTok{n_O)}\OperatorTok{/}\NormalTok{(}\OperatorTok{-}\NormalTok{q}\OperatorTok{-}\NormalTok{p}\OperatorTok{+}\DecValTok{1}\NormalTok{)}\OperatorTok{^}\DecValTok{2}
\NormalTok{    c =}\StringTok{ }\NormalTok{d2fqp    }
    
\NormalTok{    d2fqq =}\StringTok{ }\OperatorTok{-}\NormalTok{(}\DecValTok{4}\OperatorTok{*}\NormalTok{n_A)}\OperatorTok{/}\NormalTok{(}\DecValTok{2}\OperatorTok{*}\NormalTok{q}\OperatorTok{+}\NormalTok{p}\OperatorTok{-}\DecValTok{2}\NormalTok{)}\OperatorTok{^}\DecValTok{2}\OperatorTok{+}\NormalTok{n_B}\OperatorTok{*}\NormalTok{(}\OperatorTok{-}\DecValTok{1}\OperatorTok{/}\NormalTok{(q}\OperatorTok{+}\DecValTok{2}\OperatorTok{*}\NormalTok{p}\OperatorTok{-}\DecValTok{2}\NormalTok{)}\OperatorTok{^}\DecValTok{2}\OperatorTok{-}\DecValTok{1}\OperatorTok{/}\NormalTok{q}\OperatorTok{^}\DecValTok{2}\NormalTok{)}\OperatorTok{-}\NormalTok{n_AB}\OperatorTok{/}\NormalTok{q}\OperatorTok{^}\DecValTok{2}\OperatorTok{-}\NormalTok{(}\DecValTok{2}\OperatorTok{*}\NormalTok{n_O)}\OperatorTok{/}\NormalTok{(}\OperatorTok{-}\NormalTok{q}\OperatorTok{-}\NormalTok{p}\OperatorTok{+}\DecValTok{1}\NormalTok{)}\OperatorTok{^}\DecValTok{2}
\NormalTok{    d =}\StringTok{ }\NormalTok{d2fqq}
    
\NormalTok{    d2fpq =}\StringTok{ }\OperatorTok{-}\NormalTok{(}\DecValTok{2}\OperatorTok{*}\NormalTok{n_B)}\OperatorTok{/}\NormalTok{(}\DecValTok{2}\OperatorTok{*}\NormalTok{p}\OperatorTok{+}\NormalTok{q}\OperatorTok{-}\DecValTok{2}\NormalTok{)}\OperatorTok{^}\DecValTok{2}\OperatorTok{-}\NormalTok{(}\DecValTok{2}\OperatorTok{*}\NormalTok{n_A)}\OperatorTok{/}\NormalTok{(p}\OperatorTok{+}\DecValTok{2}\OperatorTok{*}\NormalTok{q}\OperatorTok{-}\DecValTok{2}\NormalTok{)}\OperatorTok{^}\DecValTok{2}\OperatorTok{-}\NormalTok{(}\DecValTok{2}\OperatorTok{*}\NormalTok{n_O)}\OperatorTok{/}\NormalTok{(}\OperatorTok{-}\NormalTok{p}\OperatorTok{-}\NormalTok{q}\OperatorTok{+}\DecValTok{1}\NormalTok{)}\OperatorTok{^}\DecValTok{2}
\NormalTok{    b =}\StringTok{ }\NormalTok{d2fpq}
    
\NormalTok{    H =}\StringTok{ }\KeywordTok{matrix}\NormalTok{( }\KeywordTok{c}\NormalTok{(a, b, c, d), }\DataTypeTok{nrow =} \DecValTok{2}\NormalTok{)}
    
    \KeywordTok{return}\NormalTok{(H)}
\NormalTok{\}}
\end{Highlighting}
\end{Shaded}

\begin{Shaded}
\begin{Highlighting}[]
\NormalTok{p_NN =}\StringTok{ }\DecValTok{0}
\NormalTok{q_NN =}\StringTok{ }\DecValTok{0}

\NormalTok{p0 =}\StringTok{ }\NormalTok{.}\DecValTok{34}
\NormalTok{q0 =}\StringTok{ }\NormalTok{.}\DecValTok{34}

\CommentTok{# p0 + q0 < 1}

\NormalTok{x0 =}\StringTok{ }\KeywordTok{c}\NormalTok{(p0, q0)}

\NormalTok{NN =}\StringTok{ }\DecValTok{100}

\CommentTok{#while( !( abs(p_N[i+1] - p_N[i]) < eps1 & abs(q_N[i+1] - q_N[i]) < eps2) | i < N) \{}

\ControlFlowTok{for}\NormalTok{ (j }\ControlFlowTok{in} \DecValTok{1}\OperatorTok{:}\NormalTok{NN) \{}

\NormalTok{  p_NN[j] =}\StringTok{ }\NormalTok{x0[}\DecValTok{1}\NormalTok{]}
\NormalTok{  q_NN[j] =}\StringTok{ }\NormalTok{x0[}\DecValTok{2}\NormalTok{]}

\NormalTok{  x1 =}\StringTok{ }\NormalTok{x0 }\OperatorTok{-}\StringTok{ }\KeywordTok{solve}\NormalTok{(}\KeywordTok{DDf}\NormalTok{(x0[}\DecValTok{1}\NormalTok{], x0[}\DecValTok{2}\NormalTok{])) }\OperatorTok\StringTok{ }\KeywordTok{Df}\NormalTok{(x0[}\DecValTok{1}\NormalTok{], x0[}\DecValTok{2}\NormalTok{])}

\CommentTok{# 2x1  2X1              2x2                      2x1}

  \CommentTok{#if( )}

\NormalTok{  x0 =}\StringTok{ }\NormalTok{x1}
  \CommentTok{#print(j)}
  \CommentTok{#print(x0)}

\NormalTok{\}}

\KeywordTok{print}\NormalTok{(x0)}
\end{Highlighting}
\end{Shaded}

\begin{verbatim}
##           [,1]
## [1,] 0.2876856
## [2,] 0.1065550
\end{verbatim}

\subsection{EM Algorithm}\label{em-algorithm}

\begin{Shaded}
\begin{Highlighting}[]
\NormalTok{n_A =}\StringTok{ }\DecValTok{9123}
\NormalTok{n_B =}\StringTok{ }\DecValTok{2987}
\NormalTok{n_AB =}\StringTok{ }\DecValTok{1269}
\NormalTok{n_O =}\StringTok{ }\DecValTok{7725}
\NormalTok{n =}\StringTok{ }\NormalTok{n_A }\OperatorTok{+}\StringTok{ }\NormalTok{n_B }\OperatorTok{+}\StringTok{ }\NormalTok{n_AB }\OperatorTok{+}\StringTok{ }\NormalTok{n_O}

\CommentTok{# Starting estimates}

\NormalTok{p_N =}\StringTok{ }\DecValTok{0}
\NormalTok{q_N =}\StringTok{ }\DecValTok{0}

\NormalTok{p =}\StringTok{ }\NormalTok{.}\DecValTok{6}
\NormalTok{q =}\StringTok{ }\NormalTok{.}\DecValTok{6}

\NormalTok{p =}\StringTok{ }\NormalTok{.}\DecValTok{34}
\NormalTok{q =}\StringTok{ }\NormalTok{.}\DecValTok{34}

\NormalTok{p_N[}\DecValTok{2}\NormalTok{] =}\StringTok{ }\NormalTok{p}
\NormalTok{q_N[}\DecValTok{2}\NormalTok{] =}\StringTok{ }\NormalTok{q}

\NormalTok{N =}\StringTok{ }\DecValTok{500} \CommentTok{# Number of max iterations.}
\NormalTok{i =}\StringTok{ }\DecValTok{1}
\NormalTok{eps1 =}\StringTok{ }\NormalTok{.}\DecValTok{00005}
\NormalTok{eps2 =}\StringTok{ }\NormalTok{.}\DecValTok{00005}

\CommentTok{# 2 assumptions: hwe and }

\ControlFlowTok{while}\NormalTok{( }\OperatorTok{!}\NormalTok{( }\KeywordTok{abs}\NormalTok{(p_N[i}\OperatorTok{+}\DecValTok{1}\NormalTok{] }\OperatorTok{-}\StringTok{ }\NormalTok{p_N[i]) }\OperatorTok{<}\StringTok{ }\NormalTok{eps1 }\OperatorTok{&}\StringTok{ }\KeywordTok{abs}\NormalTok{(q_N[i}\OperatorTok{+}\DecValTok{1}\NormalTok{] }\OperatorTok{-}\StringTok{ }\NormalTok{q_N[i]) }\OperatorTok{<}\StringTok{ }\NormalTok{eps2) }\OperatorTok{|}\StringTok{ }\NormalTok{i }\OperatorTok{<}\StringTok{ }\NormalTok{N) \{ }
      
        \CommentTok{# Expectation Step}
      
\NormalTok{        E_nAA =}\StringTok{ }\NormalTok{n_A }\OperatorTok{*}\StringTok{ }\NormalTok{( (p}\OperatorTok{^}\DecValTok{2}\NormalTok{) }\OperatorTok{/}\StringTok{ }\NormalTok{( p}\OperatorTok{^}\DecValTok{2} \OperatorTok{+}\StringTok{ }\DecValTok{2}\OperatorTok{*}\NormalTok{p}\OperatorTok{*}\NormalTok{(}\DecValTok{1}\OperatorTok{-}\NormalTok{p}\OperatorTok{-}\NormalTok{q) ) )}
\NormalTok{        E_nAO =}\StringTok{ }\NormalTok{n_A }\OperatorTok{*}\StringTok{ }\NormalTok{( (}\DecValTok{2}\OperatorTok{*}\NormalTok{p}\OperatorTok{*}\NormalTok{(}\DecValTok{1}\OperatorTok{-}\NormalTok{p}\OperatorTok{-}\NormalTok{q)) }\OperatorTok{/}\StringTok{ }\NormalTok{( p}\OperatorTok{^}\DecValTok{2} \OperatorTok{+}\StringTok{ }\DecValTok{2}\OperatorTok{*}\NormalTok{p}\OperatorTok{*}\NormalTok{(}\DecValTok{1}\OperatorTok{-}\NormalTok{p}\OperatorTok{-}\NormalTok{q) ) )}
\NormalTok{        E_nBB =}\StringTok{ }\NormalTok{n_B }\OperatorTok{*}\StringTok{ }\NormalTok{( (q}\OperatorTok{^}\DecValTok{2}\NormalTok{) }\OperatorTok{/}\StringTok{ }\NormalTok{( p}\OperatorTok{^}\DecValTok{2} \OperatorTok{+}\StringTok{ }\DecValTok{2}\OperatorTok{*}\NormalTok{p}\OperatorTok{*}\NormalTok{(}\DecValTok{1}\OperatorTok{-}\NormalTok{p}\OperatorTok{-}\NormalTok{q) ) )}
\NormalTok{        E_nBO =}\StringTok{ }\NormalTok{n_B }\OperatorTok{*}\StringTok{ }\NormalTok{( (}\DecValTok{2}\OperatorTok{*}\NormalTok{q}\OperatorTok{*}\NormalTok{(}\DecValTok{1}\OperatorTok{-}\NormalTok{p}\OperatorTok{-}\NormalTok{q)) }\OperatorTok{/}\StringTok{ }\NormalTok{( p}\OperatorTok{^}\DecValTok{2} \OperatorTok{+}\StringTok{ }\DecValTok{2}\OperatorTok{*}\NormalTok{p}\OperatorTok{*}\NormalTok{(}\DecValTok{1}\OperatorTok{-}\NormalTok{p}\OperatorTok{-}\NormalTok{q) ) )}
        
        \CommentTok{# Maximization step}
        
\NormalTok{        p =}\StringTok{ }\NormalTok{(}\DecValTok{2}\OperatorTok{*}\StringTok{ }\NormalTok{E_nAA }\OperatorTok{+}\StringTok{ }\NormalTok{E_nAO }\OperatorTok{+}\StringTok{ }\NormalTok{n_AB) }\OperatorTok{/}\StringTok{ }\NormalTok{(}\DecValTok{2}\OperatorTok{*}\NormalTok{n) }
\NormalTok{        q =}\StringTok{ }\NormalTok{(}\DecValTok{2}\OperatorTok{*}\StringTok{ }\NormalTok{E_nBB }\OperatorTok{+}\StringTok{ }\NormalTok{E_nBO }\OperatorTok{+}\StringTok{ }\NormalTok{n_AB) }\OperatorTok{/}\StringTok{ }\NormalTok{(}\DecValTok{2}\OperatorTok{*}\NormalTok{n)}
        
\NormalTok{        p_N[i}\OperatorTok{+}\DecValTok{2}\NormalTok{] =}\StringTok{ }\NormalTok{p}
\NormalTok{        q_N[i}\OperatorTok{+}\DecValTok{2}\NormalTok{] =}\StringTok{ }\NormalTok{q}
      
\NormalTok{        i =}\StringTok{ }\NormalTok{i }\OperatorTok{+}\StringTok{ }\DecValTok{1}
  
\NormalTok{\}}

\KeywordTok{print}\NormalTok{(i)}
\end{Highlighting}
\end{Shaded}

\begin{verbatim}
## [1] 500
\end{verbatim}

\begin{Shaded}
\begin{Highlighting}[]
\KeywordTok{print}\NormalTok{(p)}
\end{Highlighting}
\end{Shaded}

\begin{verbatim}
## [1] 0.2835973
\end{verbatim}

\begin{Shaded}
\begin{Highlighting}[]
\KeywordTok{print}\NormalTok{(q)}
\end{Highlighting}
\end{Shaded}

\begin{verbatim}
## [1] 0.03845067
\end{verbatim}

\section{Discussion}\label{discussion}

\subsection{Comparing Algorithm Speed and
Efficiency.}\label{comparing-algorithm-speed-and-efficiency.}

\subsection{Comparing Algorithms' robustness to initial
vaules.}\label{comparing-algorithms-robustness-to-initial-vaules.}

\subsection{}\label{section}

\section{References}\label{references}


\end{document}
