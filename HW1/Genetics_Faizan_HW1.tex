\documentclass[]{article}
\usepackage{lmodern}
\usepackage{amssymb,amsmath}
\usepackage{ifxetex,ifluatex}
\usepackage{fixltx2e} % provides \textsubscript
\ifnum 0\ifxetex 1\fi\ifluatex 1\fi=0 % if pdftex
  \usepackage[T1]{fontenc}
  \usepackage[utf8]{inputenc}
\else % if luatex or xelatex
  \ifxetex
    \usepackage{mathspec}
  \else
    \usepackage{fontspec}
  \fi
  \defaultfontfeatures{Ligatures=TeX,Scale=MatchLowercase}
\fi
% use upquote if available, for straight quotes in verbatim environments
\IfFileExists{upquote.sty}{\usepackage{upquote}}{}
% use microtype if available
\IfFileExists{microtype.sty}{%
\usepackage{microtype}
\UseMicrotypeSet[protrusion]{basicmath} % disable protrusion for tt fonts
}{}
\usepackage[margin=1in]{geometry}
\usepackage{hyperref}
\hypersetup{unicode=true,
            pdftitle={Statistical Genetics HW1},
            pdfauthor={Faizan Khalid Mohsin},
            pdfborder={0 0 0},
            breaklinks=true}
\urlstyle{same}  % don't use monospace font for urls
\usepackage{color}
\usepackage{fancyvrb}
\newcommand{\VerbBar}{|}
\newcommand{\VERB}{\Verb[commandchars=\\\{\}]}
\DefineVerbatimEnvironment{Highlighting}{Verbatim}{commandchars=\\\{\}}
% Add ',fontsize=\small' for more characters per line
\usepackage{framed}
\definecolor{shadecolor}{RGB}{248,248,248}
\newenvironment{Shaded}{\begin{snugshade}}{\end{snugshade}}
\newcommand{\KeywordTok}[1]{\textcolor[rgb]{0.13,0.29,0.53}{\textbf{#1}}}
\newcommand{\DataTypeTok}[1]{\textcolor[rgb]{0.13,0.29,0.53}{#1}}
\newcommand{\DecValTok}[1]{\textcolor[rgb]{0.00,0.00,0.81}{#1}}
\newcommand{\BaseNTok}[1]{\textcolor[rgb]{0.00,0.00,0.81}{#1}}
\newcommand{\FloatTok}[1]{\textcolor[rgb]{0.00,0.00,0.81}{#1}}
\newcommand{\ConstantTok}[1]{\textcolor[rgb]{0.00,0.00,0.00}{#1}}
\newcommand{\CharTok}[1]{\textcolor[rgb]{0.31,0.60,0.02}{#1}}
\newcommand{\SpecialCharTok}[1]{\textcolor[rgb]{0.00,0.00,0.00}{#1}}
\newcommand{\StringTok}[1]{\textcolor[rgb]{0.31,0.60,0.02}{#1}}
\newcommand{\VerbatimStringTok}[1]{\textcolor[rgb]{0.31,0.60,0.02}{#1}}
\newcommand{\SpecialStringTok}[1]{\textcolor[rgb]{0.31,0.60,0.02}{#1}}
\newcommand{\ImportTok}[1]{#1}
\newcommand{\CommentTok}[1]{\textcolor[rgb]{0.56,0.35,0.01}{\textit{#1}}}
\newcommand{\DocumentationTok}[1]{\textcolor[rgb]{0.56,0.35,0.01}{\textbf{\textit{#1}}}}
\newcommand{\AnnotationTok}[1]{\textcolor[rgb]{0.56,0.35,0.01}{\textbf{\textit{#1}}}}
\newcommand{\CommentVarTok}[1]{\textcolor[rgb]{0.56,0.35,0.01}{\textbf{\textit{#1}}}}
\newcommand{\OtherTok}[1]{\textcolor[rgb]{0.56,0.35,0.01}{#1}}
\newcommand{\FunctionTok}[1]{\textcolor[rgb]{0.00,0.00,0.00}{#1}}
\newcommand{\VariableTok}[1]{\textcolor[rgb]{0.00,0.00,0.00}{#1}}
\newcommand{\ControlFlowTok}[1]{\textcolor[rgb]{0.13,0.29,0.53}{\textbf{#1}}}
\newcommand{\OperatorTok}[1]{\textcolor[rgb]{0.81,0.36,0.00}{\textbf{#1}}}
\newcommand{\BuiltInTok}[1]{#1}
\newcommand{\ExtensionTok}[1]{#1}
\newcommand{\PreprocessorTok}[1]{\textcolor[rgb]{0.56,0.35,0.01}{\textit{#1}}}
\newcommand{\AttributeTok}[1]{\textcolor[rgb]{0.77,0.63,0.00}{#1}}
\newcommand{\RegionMarkerTok}[1]{#1}
\newcommand{\InformationTok}[1]{\textcolor[rgb]{0.56,0.35,0.01}{\textbf{\textit{#1}}}}
\newcommand{\WarningTok}[1]{\textcolor[rgb]{0.56,0.35,0.01}{\textbf{\textit{#1}}}}
\newcommand{\AlertTok}[1]{\textcolor[rgb]{0.94,0.16,0.16}{#1}}
\newcommand{\ErrorTok}[1]{\textcolor[rgb]{0.64,0.00,0.00}{\textbf{#1}}}
\newcommand{\NormalTok}[1]{#1}
\usepackage{graphicx,grffile}
\makeatletter
\def\maxwidth{\ifdim\Gin@nat@width>\linewidth\linewidth\else\Gin@nat@width\fi}
\def\maxheight{\ifdim\Gin@nat@height>\textheight\textheight\else\Gin@nat@height\fi}
\makeatother
% Scale images if necessary, so that they will not overflow the page
% margins by default, and it is still possible to overwrite the defaults
% using explicit options in \includegraphics[width, height, ...]{}
\setkeys{Gin}{width=\maxwidth,height=\maxheight,keepaspectratio}
\IfFileExists{parskip.sty}{%
\usepackage{parskip}
}{% else
\setlength{\parindent}{0pt}
\setlength{\parskip}{6pt plus 2pt minus 1pt}
}
\setlength{\emergencystretch}{3em}  % prevent overfull lines
\providecommand{\tightlist}{%
  \setlength{\itemsep}{0pt}\setlength{\parskip}{0pt}}
\setcounter{secnumdepth}{0}
% Redefines (sub)paragraphs to behave more like sections
\ifx\paragraph\undefined\else
\let\oldparagraph\paragraph
\renewcommand{\paragraph}[1]{\oldparagraph{#1}\mbox{}}
\fi
\ifx\subparagraph\undefined\else
\let\oldsubparagraph\subparagraph
\renewcommand{\subparagraph}[1]{\oldsubparagraph{#1}\mbox{}}
\fi

%%% Use protect on footnotes to avoid problems with footnotes in titles
\let\rmarkdownfootnote\footnote%
\def\footnote{\protect\rmarkdownfootnote}

%%% Change title format to be more compact
\usepackage{titling}

% Create subtitle command for use in maketitle
\providecommand{\subtitle}[1]{
  \posttitle{
    \begin{center}\large#1\end{center}
    }
}

\setlength{\droptitle}{-2em}

  \title{Statistical Genetics HW1}
    \pretitle{\vspace{\droptitle}\centering\huge}
  \posttitle{\par}
    \author{Faizan Khalid Mohsin}
    \preauthor{\centering\large\emph}
  \postauthor{\par}
      \predate{\centering\large\emph}
  \postdate{\par}
    \date{September 22, 2020}


\begin{document}
\maketitle

\section{Introduction}\label{introduction}

\section{Method}\label{method}

We use two methods for approximating the allele frequencies:
Expectation-Maximization method and Newton-Raphson method.

\subsection{Expectation-Maximization}\label{expectation-maximization}

To implement the EM algorithm we reframe the question in terms of a
missing data problem.

\subsection{Newton-Raphson}\label{newton-raphson}

To approximate the values of

\section{Analysis and Results.}\label{analysis-and-results.}

\subsection{Newton-Raphson Algorithm}\label{newton-raphson-algorithm}

\begin{Shaded}
\begin{Highlighting}[]
\CommentTok{# Data}

\NormalTok{n_A =}\StringTok{ }\DecValTok{9123}
\NormalTok{n_B =}\StringTok{ }\DecValTok{2987}
\NormalTok{n_AB =}\StringTok{ }\DecValTok{1269}
\NormalTok{n_O =}\StringTok{ }\DecValTok{7725}
\NormalTok{n =}\StringTok{ }\NormalTok{n_A }\OperatorTok{+}\StringTok{ }\NormalTok{n_B }\OperatorTok{+}\StringTok{ }\NormalTok{n_AB }\OperatorTok{+}\StringTok{ }\NormalTok{n_O}
\end{Highlighting}
\end{Shaded}

\begin{Shaded}
\begin{Highlighting}[]
\CommentTok{# The log likelihood function,}

\NormalTok{l =}\StringTok{ }\ControlFlowTok{function}\NormalTok{(p, q) \{}

\NormalTok{  f =}\StringTok{ }\NormalTok{n_A }\OperatorTok{*}\StringTok{ }\KeywordTok{log}\NormalTok{(p}\OperatorTok{^}\DecValTok{2}\OperatorTok{+}\DecValTok{2}\OperatorTok{*}\NormalTok{p}\OperatorTok{*}\NormalTok{(}\DecValTok{1}\NormalTok{−p−q)) }\OperatorTok{+}\StringTok{ }\NormalTok{n_B}\OperatorTok{*}\KeywordTok{log}\NormalTok{(q}\OperatorTok{^}\DecValTok{2}\OperatorTok{+}\DecValTok{2}\OperatorTok{*}\NormalTok{q}\OperatorTok{*}\NormalTok{(}\DecValTok{1}\NormalTok{−p−q)) }\OperatorTok{+}\StringTok{ }\NormalTok{n_AB}\OperatorTok{*}\KeywordTok{log}\NormalTok{(}\DecValTok{2}\OperatorTok{*}\NormalTok{p}\OperatorTok{*}\NormalTok{q) }\OperatorTok{+}\StringTok{ }\NormalTok{n_O}\OperatorTok{*}\KeywordTok{log}\NormalTok{((}\DecValTok{1}\NormalTok{−p−q)}\OperatorTok{^}\DecValTok{2}\NormalTok{)}

  \KeywordTok{return}\NormalTok{(f)}
\NormalTok{\}}



\CommentTok{# Function that calculates the full first derivative. }

\NormalTok{Df =}\StringTok{ }\ControlFlowTok{function}\NormalTok{(p, q)\{}
  
\NormalTok{  dfp =}\StringTok{   }\NormalTok{( n_A}\OperatorTok{*}\NormalTok{(}\DecValTok{2} \OperatorTok{-}\StringTok{ }\DecValTok{2}\OperatorTok{*}\NormalTok{p }\OperatorTok{-}\StringTok{ }\DecValTok{2}\OperatorTok{*}\NormalTok{q) }\OperatorTok{/}\StringTok{ }\NormalTok{(p}\OperatorTok{^}\DecValTok{2}\OperatorTok{+}\DecValTok{2}\OperatorTok{*}\NormalTok{p}\OperatorTok{*}\NormalTok{(}\DecValTok{1}\NormalTok{−p−q)) )   }
        \OperatorTok{+}\StringTok{ }\NormalTok{(n_B }\OperatorTok{*}\StringTok{ }\NormalTok{(}\OperatorTok{-}\DecValTok{2}\OperatorTok{*}\NormalTok{q) }\OperatorTok{/}\StringTok{ }\NormalTok{(q}\OperatorTok{^}\DecValTok{2}\OperatorTok{+}\DecValTok{2}\OperatorTok{*}\NormalTok{q}\OperatorTok{*}\NormalTok{(}\DecValTok{1}\NormalTok{−p−q))) }
        \OperatorTok{+}\StringTok{ }\NormalTok{(n_AB}\OperatorTok{/}\NormalTok{p) }
        \OperatorTok{+}\StringTok{ }\NormalTok{(}\OperatorTok{-}\DecValTok{2}\OperatorTok{*}\NormalTok{n_O }\OperatorTok{/}\StringTok{ }\NormalTok{(}\DecValTok{1}\NormalTok{−p−q))}
  
\NormalTok{  dfq =}\StringTok{   }\NormalTok{( n_A}\OperatorTok{*}\NormalTok{(}\DecValTok{2}\OperatorTok{*}\NormalTok{p ) }\OperatorTok{/}\StringTok{ }\NormalTok{(p}\OperatorTok{^}\DecValTok{2}\OperatorTok{+}\DecValTok{2}\OperatorTok{*}\NormalTok{p}\OperatorTok{*}\NormalTok{(}\DecValTok{1}\NormalTok{−p−q)) )   }
        \OperatorTok{+}\StringTok{ }\NormalTok{(n_B }\OperatorTok{*}\StringTok{ }\NormalTok{(}\DecValTok{2} \OperatorTok{-}\StringTok{ }\DecValTok{2}\OperatorTok{*}\NormalTok{p }\OperatorTok{-}\StringTok{ }\DecValTok{2}\OperatorTok{*}\NormalTok{q) }\OperatorTok{/}\StringTok{ }\NormalTok{(q}\OperatorTok{^}\DecValTok{2}\OperatorTok{+}\DecValTok{2}\OperatorTok{*}\NormalTok{q}\OperatorTok{*}\NormalTok{(}\DecValTok{1}\NormalTok{−p−q))) }
        \OperatorTok{+}\StringTok{ }\NormalTok{(n_AB}\OperatorTok{/}\NormalTok{q) }
        \OperatorTok{+}\StringTok{ }\NormalTok{(}\OperatorTok{-}\DecValTok{2}\OperatorTok{*}\NormalTok{n_O }\OperatorTok{/}\StringTok{ }\NormalTok{(}\DecValTok{1}\NormalTok{−p−q))}
  
  \KeywordTok{return}\NormalTok{( }\KeywordTok{c}\NormalTok{(dfp, dfq) )}
\NormalTok{\}}



\CommentTok{# Function that calculates the Hessian.}

\NormalTok{DDf =}\StringTok{ }\ControlFlowTok{function}\NormalTok{(p, q) \{}
  
\NormalTok{    d2fpp =}\StringTok{ }\NormalTok{n_A }\OperatorTok{*}\StringTok{ }\KeywordTok{log}\NormalTok{(p}\OperatorTok{^}\DecValTok{2}\OperatorTok{+}\DecValTok{2}\OperatorTok{*}\NormalTok{p}\OperatorTok{*}\NormalTok{(}\DecValTok{1}\NormalTok{−p−q)) }\OperatorTok{+}\StringTok{ }\NormalTok{n_B}\OperatorTok{*}\KeywordTok{log}\NormalTok{(q}\OperatorTok{^}\DecValTok{2}\OperatorTok{+}\DecValTok{2}\OperatorTok{*}\NormalTok{q}\OperatorTok{*}\NormalTok{(}\DecValTok{1}\NormalTok{−p−q)) }\OperatorTok{+}\StringTok{ }\NormalTok{n_AB}\OperatorTok{*}\KeywordTok{log}\NormalTok{(}\DecValTok{2}\OperatorTok{*}\NormalTok{p}\OperatorTok{*}\NormalTok{q) }\OperatorTok{+}\StringTok{ }\NormalTok{n_O}\OperatorTok{*}\KeywordTok{log}\NormalTok{((}\DecValTok{1}\NormalTok{−p−q)}\OperatorTok{^}\DecValTok{2}\NormalTok{) }

\NormalTok{    a =}\StringTok{ }\NormalTok{d2fpp}
    
\NormalTok{    d2fqq =}\StringTok{ }\NormalTok{n_A }\OperatorTok{*}\StringTok{ }\KeywordTok{log}\NormalTok{(p}\OperatorTok{^}\DecValTok{2}\OperatorTok{+}\DecValTok{2}\OperatorTok{*}\NormalTok{p}\OperatorTok{*}\NormalTok{(}\DecValTok{1}\NormalTok{−p−q)) }\OperatorTok{+}\StringTok{ }\NormalTok{n_B}\OperatorTok{*}\KeywordTok{log}\NormalTok{(q}\OperatorTok{^}\DecValTok{2}\OperatorTok{+}\DecValTok{2}\OperatorTok{*}\NormalTok{q}\OperatorTok{*}\NormalTok{(}\DecValTok{1}\NormalTok{−p−q)) }\OperatorTok{+}\StringTok{ }\NormalTok{n_AB}\OperatorTok{*}\KeywordTok{log}\NormalTok{(}\DecValTok{2}\OperatorTok{*}\NormalTok{p}\OperatorTok{*}\NormalTok{q) }\OperatorTok{+}\StringTok{ }\NormalTok{n_O}\OperatorTok{*}\KeywordTok{log}\NormalTok{((}\DecValTok{1}\NormalTok{−p−q)}\OperatorTok{^}\DecValTok{2}\NormalTok{)}

    
\NormalTok{     d =}\StringTok{ }\NormalTok{d2fqq}
    
\NormalTok{    d2fpq =}\StringTok{ }\NormalTok{n_A }\OperatorTok{*}\StringTok{ }\KeywordTok{log}\NormalTok{(p}\OperatorTok{^}\DecValTok{2}\OperatorTok{+}\DecValTok{2}\OperatorTok{*}\NormalTok{p}\OperatorTok{*}\NormalTok{(}\DecValTok{1}\NormalTok{−p−q)) }\OperatorTok{+}\StringTok{ }\NormalTok{n_B}\OperatorTok{*}\KeywordTok{log}\NormalTok{(q}\OperatorTok{^}\DecValTok{2}\OperatorTok{+}\DecValTok{2}\OperatorTok{*}\NormalTok{q}\OperatorTok{*}\NormalTok{(}\DecValTok{1}\NormalTok{−p−q)) }\OperatorTok{+}\StringTok{ }\NormalTok{n_AB}\OperatorTok{*}\KeywordTok{log}\NormalTok{(}\DecValTok{2}\OperatorTok{*}\NormalTok{p}\OperatorTok{*}\NormalTok{q) }\OperatorTok{+}\StringTok{ }\NormalTok{n_O}\OperatorTok{*}\KeywordTok{log}\NormalTok{((}\DecValTok{1}\NormalTok{−p−q)}\OperatorTok{^}\DecValTok{2}\NormalTok{) }
    
\NormalTok{     b =}\StringTok{ }\NormalTok{d2fpq}

    
\NormalTok{    d2fqp =}\StringTok{ }\NormalTok{n_A }\OperatorTok{*}\StringTok{ }\KeywordTok{log}\NormalTok{(p}\OperatorTok{^}\DecValTok{2}\OperatorTok{+}\DecValTok{2}\OperatorTok{*}\NormalTok{p}\OperatorTok{*}\NormalTok{(}\DecValTok{1}\NormalTok{−p−q)) }\OperatorTok{+}\StringTok{ }\NormalTok{n_B}\OperatorTok{*}\KeywordTok{log}\NormalTok{(q}\OperatorTok{^}\DecValTok{2}\OperatorTok{+}\DecValTok{2}\OperatorTok{*}\NormalTok{q}\OperatorTok{*}\NormalTok{(}\DecValTok{1}\NormalTok{−p−q)) }\OperatorTok{+}\StringTok{ }\NormalTok{n_AB}\OperatorTok{*}\KeywordTok{log}\NormalTok{(}\DecValTok{2}\OperatorTok{*}\NormalTok{p}\OperatorTok{*}\NormalTok{q) }\OperatorTok{+}\StringTok{ }\NormalTok{n_O}\OperatorTok{*}\KeywordTok{log}\NormalTok{((}\DecValTok{1}\NormalTok{−p−q)}\OperatorTok{^}\DecValTok{2}\NormalTok{) }
    
\NormalTok{     c =}\StringTok{ }\NormalTok{d2fqp}

\NormalTok{    H =}\StringTok{ }\KeywordTok{matrix}\NormalTok{( }\KeywordTok{c}\NormalTok{(a, b, c, d), }\DataTypeTok{nrow =} \DecValTok{2}\NormalTok{)}
\NormalTok{\}}

\CommentTok{#H = 2 x 2}
\end{Highlighting}
\end{Shaded}

\begin{Shaded}
\begin{Highlighting}[]
\CommentTok{# p_NN}
\CommentTok{# q_NN}
\CommentTok{# }
\CommentTok{# p0 = .1}
\CommentTok{# q0 = .2}
\CommentTok{# }
\CommentTok{# NN = 100}
\CommentTok{# }
\CommentTok{# #while( !( abs(p_N[i+1] - p_N[i]) < eps1 & abs(q_N[i+1] - q_N[i]) < eps2) | i < N) \{}
\CommentTok{# }
\CommentTok{# for (i in 1:NN) \{}
\CommentTok{# }
\CommentTok{#   p_NN[i] = p0}
\CommentTok{#   q_NN[i] = q0}
\CommentTok{# }
\CommentTok{#   p1 = p0 - DDl(p0, q0) %*% Dl(p0, q0)  }
\CommentTok{#   }
\CommentTok{#    2x1     2X1          2x2          2x1}
\CommentTok{# }
\CommentTok{# }
\CommentTok{# }
\CommentTok{#   #if( )}
\CommentTok{# }
\CommentTok{# }
\CommentTok{#   p0 = p1}
\CommentTok{#   q0 = q1}
\CommentTok{# }
\CommentTok{# }
\CommentTok{# }
\CommentTok{# }
\CommentTok{# }
\CommentTok{# }
\CommentTok{# \}}
\end{Highlighting}
\end{Shaded}

\subsection{EM Algorithm}\label{em-algorithm}

\begin{Shaded}
\begin{Highlighting}[]
\NormalTok{n_A =}\StringTok{ }\DecValTok{9123}
\NormalTok{n_B =}\StringTok{ }\DecValTok{2987}
\NormalTok{n_AB =}\StringTok{ }\DecValTok{1269}
\NormalTok{n_O =}\StringTok{ }\DecValTok{7725}
\NormalTok{n =}\StringTok{ }\NormalTok{n_A }\OperatorTok{+}\StringTok{ }\NormalTok{n_B }\OperatorTok{+}\StringTok{ }\NormalTok{n_AB }\OperatorTok{+}\StringTok{ }\NormalTok{n_O}


\CommentTok{# Starting estimates}

\NormalTok{p_N =}\StringTok{ }\DecValTok{0}
\NormalTok{q_N =}\StringTok{ }\DecValTok{0}
        
\NormalTok{E_nAA_N =}\StringTok{ }\DecValTok{0}
\NormalTok{E_nAO_N =}\StringTok{ }\DecValTok{0}
\NormalTok{E_nBB_N =}\StringTok{ }\DecValTok{0}
\NormalTok{E_nBO_N =}\StringTok{ }\DecValTok{0}

\NormalTok{p_N[}\DecValTok{2}\NormalTok{] =}\StringTok{ }\NormalTok{.}\DecValTok{6}
\NormalTok{q_N[}\DecValTok{2}\NormalTok{] =}\StringTok{ }\NormalTok{.}\DecValTok{6}
        
\CommentTok{# E_nAA_N[2] = .5}
\CommentTok{# E_nAO_N[2] = .5}
\CommentTok{# E_nBB_N[2] = .5}
\CommentTok{# E_nBO_N[2] = .5}

\CommentTok{#for ( i in 1:N) \{}
\NormalTok{N =}\StringTok{ }\DecValTok{100} \CommentTok{# Number of max iterations.}
\NormalTok{i =}\StringTok{ }\DecValTok{1}
\NormalTok{p =}\StringTok{ }\NormalTok{.}\DecValTok{6}
\NormalTok{q =}\StringTok{ }\NormalTok{.}\DecValTok{6}
\NormalTok{eps1 =}\StringTok{ }\NormalTok{.}\DecValTok{00005}
\NormalTok{eps2 =}\StringTok{ }\NormalTok{.}\DecValTok{00005}
\CommentTok{# 2 assumptions }

\ControlFlowTok{while}\NormalTok{( }\OperatorTok{!}\NormalTok{( }\KeywordTok{abs}\NormalTok{(p_N[i}\OperatorTok{+}\DecValTok{1}\NormalTok{] }\OperatorTok{-}\StringTok{ }\NormalTok{p_N[i]) }\OperatorTok{<}\StringTok{ }\NormalTok{eps1 }\OperatorTok{&}\StringTok{ }\KeywordTok{abs}\NormalTok{(q_N[i}\OperatorTok{+}\DecValTok{1}\NormalTok{] }\OperatorTok{-}\StringTok{ }\NormalTok{q_N[i]) }\OperatorTok{<}\StringTok{ }\NormalTok{eps2) }\OperatorTok{|}\StringTok{ }\NormalTok{i }\OperatorTok{<}\StringTok{ }\NormalTok{N) \{ }
      
        \CommentTok{# Expectation Step}
      
\NormalTok{        E_nAA =}\StringTok{ }\NormalTok{n_A }\OperatorTok{*}\StringTok{ }\NormalTok{( (p}\OperatorTok{^}\DecValTok{2}\NormalTok{) }\OperatorTok{/}\StringTok{ }\NormalTok{( p}\OperatorTok{^}\DecValTok{2} \OperatorTok{+}\StringTok{ }\DecValTok{2}\OperatorTok{*}\NormalTok{p}\OperatorTok{*}\NormalTok{(}\DecValTok{1}\OperatorTok{-}\NormalTok{p}\OperatorTok{-}\NormalTok{q) ) )}
\NormalTok{        E_nAO =}\StringTok{ }\NormalTok{n_A }\OperatorTok{*}\StringTok{ }\NormalTok{( (}\DecValTok{2}\OperatorTok{*}\NormalTok{p}\OperatorTok{*}\NormalTok{(}\DecValTok{1}\OperatorTok{-}\NormalTok{p}\OperatorTok{-}\NormalTok{q)) }\OperatorTok{/}\StringTok{ }\NormalTok{( p}\OperatorTok{^}\DecValTok{2} \OperatorTok{+}\StringTok{ }\DecValTok{2}\OperatorTok{*}\NormalTok{p}\OperatorTok{*}\NormalTok{(}\DecValTok{1}\OperatorTok{-}\NormalTok{p}\OperatorTok{-}\NormalTok{q) ) )}
\NormalTok{        E_nBB =}\StringTok{ }\NormalTok{n_B }\OperatorTok{*}\StringTok{ }\NormalTok{( (q}\OperatorTok{^}\DecValTok{2}\NormalTok{) }\OperatorTok{/}\StringTok{ }\NormalTok{( p}\OperatorTok{^}\DecValTok{2} \OperatorTok{+}\StringTok{ }\DecValTok{2}\OperatorTok{*}\NormalTok{p}\OperatorTok{*}\NormalTok{(}\DecValTok{1}\OperatorTok{-}\NormalTok{p}\OperatorTok{-}\NormalTok{q) ) )}
\NormalTok{        E_nBO =}\StringTok{ }\NormalTok{n_B }\OperatorTok{*}\StringTok{ }\NormalTok{( (}\DecValTok{2}\OperatorTok{*}\NormalTok{q}\OperatorTok{*}\NormalTok{(}\DecValTok{1}\OperatorTok{-}\NormalTok{p}\OperatorTok{-}\NormalTok{q)) }\OperatorTok{/}\StringTok{ }\NormalTok{( p}\OperatorTok{^}\DecValTok{2} \OperatorTok{+}\StringTok{ }\DecValTok{2}\OperatorTok{*}\NormalTok{p}\OperatorTok{*}\NormalTok{(}\DecValTok{1}\OperatorTok{-}\NormalTok{p}\OperatorTok{-}\NormalTok{q) ) )}
        
        
        \CommentTok{# Maximization step}
        
\NormalTok{        p =}\StringTok{ }\NormalTok{(}\DecValTok{2}\OperatorTok{*}\StringTok{ }\NormalTok{E_nAA }\OperatorTok{+}\StringTok{ }\NormalTok{E_nAO }\OperatorTok{+}\StringTok{ }\NormalTok{n_AB) }\OperatorTok{/}\StringTok{ }\NormalTok{(}\DecValTok{2}\OperatorTok{*}\NormalTok{n) }
\NormalTok{        q =}\StringTok{ }\NormalTok{(}\DecValTok{2}\OperatorTok{*}\StringTok{ }\NormalTok{E_nBB }\OperatorTok{+}\StringTok{ }\NormalTok{E_nBO }\OperatorTok{+}\StringTok{ }\NormalTok{n_AB) }\OperatorTok{/}\StringTok{ }\NormalTok{(}\DecValTok{2}\OperatorTok{*}\NormalTok{n)}
        
        \CommentTok{# print(paste("i: ", i))}
        \CommentTok{# print(paste("p:", p, "q:", q))}
        
        \CommentTok{# Store the values}
        
\NormalTok{        p_N[i}\OperatorTok{+}\DecValTok{2}\NormalTok{] =}\StringTok{ }\NormalTok{p}
\NormalTok{        q_N[i}\OperatorTok{+}\DecValTok{2}\NormalTok{] =}\StringTok{ }\NormalTok{q}
        
        \CommentTok{# E_nAA_N[i+2] = E_nAA}
        \CommentTok{# E_nAO_N[i+2] = E_nAO}
        \CommentTok{# E_nBB_N[i+2] = E_nBB}
        \CommentTok{# E_nBO_N[i+2] = E_nBO}
        
        \CommentTok{#df = data.frame()}
        
\NormalTok{        i =}\StringTok{ }\NormalTok{i }\OperatorTok{+}\StringTok{ }\DecValTok{1}
  
\NormalTok{\}}


\KeywordTok{print}\NormalTok{(p)}
\end{Highlighting}
\end{Shaded}

\begin{verbatim}
## [1] 0.2835973
\end{verbatim}

\begin{Shaded}
\begin{Highlighting}[]
\KeywordTok{print}\NormalTok{(q)}
\end{Highlighting}
\end{Shaded}

\begin{verbatim}
## [1] 0.03845067
\end{verbatim}

\section{Discussion}\label{discussion}

\subsection{Comparing Algorithm Speed and
Efficiency.}\label{comparing-algorithm-speed-and-efficiency.}


\end{document}
